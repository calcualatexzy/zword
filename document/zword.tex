\documentclass{./source/Report}

\name{许子绎、王励劼、陈科}
\date{\today}
\course{面向对象程序设计}
\instructor{陈奇}
\expname{文本编辑器}%实验名称

\begin{document}

\makecover

\tableofcontents

\newpage

\section{需求分析}

\subsection{要求}
实现文本编辑器
\subsection{功能}
\begin{enumerate}
    \item 支持基本的文本编辑功能,包括输入、回退、插入、删除、撤销、重做等
    \item 支持文件的保存和打开
    \item 支持搜索与替换
    \item 支持文字格式的设置,包括大小、字体、颜色等
    \item 支持排版设置,包括对齐方式、自动分页等
\end{enumerate}

\section{总体设计}

% 实现关键点
% \begin{itemize}
%     \item 保存文档的数据结构
%     \item 如何处理连续输入文档时的内存分配情况,一次分配足够大还是每次都申请一段内存
%     \item 多级的撤销与重做,如何记录对文挡每次的操作
%     \item 如何表示文档中不同的文字格式
%     \item 如何记录段落格式
%     \item 支持多页文档的时候,如何进行分页
%     \item 汉字的储存和显示
% \end{itemize}

\section{系统模块说明}
%每一部分应有核心类说明

\subsection{界面}

\subsection{存储}

\section{设计难点与解决方案}

\section{总结}

\newpage
\appendix
\section{程序使用说明}

\section{项目开发日志}

\subsection{开发准备}
\subsubsection{环境配置}
使用 Qt6.5.0 配合 Qt Creator 或 VS2022 作为开发环境,\href{https://blog.csdn.net/m0_62919535/article/details/129340079}{配置 VS + Qt 的参考教程}。

\subsubsection{合作方式}
使用 github 开设私人仓库并设置合作者实现合作开发,可能使用 PR (Pull Requset) 实现更好的整合。

\subsubsection{文档写著}
使用 \LaTeX 进行文档的书写。

\end{document}